\chapter{Introduzione}
Si definisca la logica attraverso il suo oggetto principale e il suo scopo: la logica \`e la disciplina che studia le condizioni di correttezza del ragionamento. Il suo scopo \`e elaborare criteri e metodi attraverso i 
quali si possano distinguere i ragionamenti corretti o validi da quelli incorretti o invalidi. Si considera un argomento un gruppo strutturato di enunciati. \`E pertanto costituito da un insieme di enunciati detti 
premesse dati per certi che vengono asserite a sostegno di un enunciato conclusione. Esistono espressioni dette indicatori di conclusione che vengono utilizzate per segnalare il passaggio dalle premesse alla 
conclusione definita come l'enunciato che viene affermato sulla base delle premesse, mentre queste ultime come enunciati asseriti o assunti che forniscono il fondamento per affermare la conclusione. Quando 
ci\`o accade si dice che la conclusione segue dalle premesse. Indagare sulla validit\`a di un ragionamento vuol dire pertanto indagare se la sua conclusione segue effettivamente dalle premesse che ne 
costituirebbero un buon fondamento. Si definisce inferenza il processo con cui si giunge ad accettare la conclusione di un ragionamento sulla base delle sue premesse. Si pu\`o pertanto dire che la logica studia le 
condizioni di correttezza delle inferenze. Si definiscono argomentazioni complesse argomentazioni in cui una conclusione di una precedenza inferenza viene utilizzata come premessa in una successiva. 
\section{Enunciati dichiarativi}
Gli enunciati dichiarativi sono gli enunciati caratteristici di un ragionamento. Si definisce enunciato dichiarativo pertanto un discorso con la propriet\`a di essere univocamente vero o falso. Una proposizione \`e il 
senso di un enunciato, che la esprime. Si riconosce un enunciato in quanto equivale a un gruppo di valore a cui \`e possibile assegnare il valore di verit\`a indipendentemente dal resto. 
\section{Essere corretti, essere veridici}
Se verit\`a e falsit\`a sono caratteristiche proprie degli enunciati, la correttezza caratterizza i ragionamenti. Si pu\`o dire che la correttezza \`e una propriet\`a delle relazioni che sussistono tra enunciati. Dalla 
verit\`a degli enunciati non segue n\`e necessariamente n\`e sufficientemente la correttezza del ragionamento. Il seguire di una conclusione dalle sue premesse \`e caratterizzato dal criterio generale di 
correttezza logica: un ragionamento \`e corretto se e solo se non pu\`o darsi il caso che le sue premesse siano tutte vere e la sua conclusione falsa. Si dice che un enunciato $P$ \`e conseguenza logica di altri 
enunciati $P_1,\dots, P_n$ se e solo se in ogni circostanza o situazione in cui $P_1,\dots, P_n$ sono veri anche $P$ \`e vero, ovvero che se e solo se $P_1,\dots, P_n$ siano veri mentre $P$ \`e vero. Si pu\`o 
pertanto dire come un ragionamento \`e corretto se e solo se la sua conclusione \`e conseguenza logica delle sue premesse. 
\section{Induzione e deduzione}
Per descrivere induzione e deduzione si devono innanzitutto caratterizzare gli enunciati in universali, particolari e singolari. Si dicono universali tutti gli enunciati che riguardano un'intera classe di individui, 
particolari gli enunciati che riguardano alcuni elementi di una classe e singolari tutti quelli che caratterizzano un unico elemento. Sia per ragionamenti induttivi che deduttivi si dice che le loro premesse devono 
fornire un fondamento per l'affermazione della conclusione. Vengono caratterizzati dal modo in cui si arriva a tale conclusione: in un ragionamento induttivo le premesse non forniscono ragioni decisive per la 
conclusione ma si limitano a garantirla secondo diversi gradi di probabilit\`a, mentre per un ragionamento deduttivo la correttezza \`e determinata dal criterio di correttezza logica e fornisce pertanto un 
fondamento infallibile e definitivo per la verit\`a. 
\section{Forma logica}
Trattando di logica formale si rende necessario esprimere una forma logica dei ragionamenti per verificarne le condizioni di validit\`a. Si pu\`o banalmente notare come ragionamenti di contesto diverso possono 
avere forme comuni e si possono pertanto trarre delle osservazioni. 
\subsubsection{Osservazioni}
La correttezza dei ragionamenti dipende dalla forma o struttura logica che pu\`o essere studiata indipendentemente dal contenuto degli enunciati. Inoltre la verit\`a degli enunciati riguarda il contenuto 
caratterizzato dagli enunciati che non essendo studio della logica verr\`a tralasciato. In fine si pu\`o notare come la forma sia pi\`u generale del contenuto, ovvero sostituendo ad una forma enunciati di qualsiasi 
contenuto si otterr\`a un ragionamento valido.
\section{Logica e scienze particolari}
La logica astrae largamente dal contenuto e si pu\`o dire che studi le leggi dell'esser vero. Si dice pertanto che la logica prescinde dall'accertamento della verit\`a degli enunciati. Si occupa delle relazioni logiche 
che intercorrono fra gli enunciati: fra premesse e conclusioni e pertanto della validit\`a delle inferenze ed \`e pertanto presente in ogni scienza. 
\section{Parole logiche}
Essendo che la validit\`a degli schemi logici non dipende dal significato particolare delle espressioni dipende da altre espressioni costanti che si dicono descrittive che permettono di legare gli enunciati o 
enumerarli.
\section{Diagrammi}
Ai fini dell'analisi logica \`e utile rappresentare la struttura di un'argomentazione in modo da evidenziarne i nessi inferenziali in maniera schematica. A questo scopo ogni asserzione viene numerata. Se due o 
pi\`u premesse operano congiuntamente in un passo induttivo si trascrivono i loro numeri uniti dal segno $+$ e sottolineate. La conclusione viene descritta da una freccia verticale dai numeri delle premesse a 
quello della stessa. L'intera procedura \`e ripetuta per ogni passo dell'argomentazione. 
\section{Argomentazioni convergenti}
Si dice argomentazione convergente un'argomentazione che contiene diversi passi di ragionamento separati e ciascuno sostiene la stessa conclusione. 
\section{Asserzioni implicite}
\`E possibile che alcuni enunciati non siano esplicitamente dichiarati durante l'argomentazione ma lasciati impliciti in quanto di banale comprensione. 
\section{Valutare un'argomentazione}
\subsection{Criteri di valutazione}
Per determinare se un'argomentazione dimostra la verit\`a della conclusione vengono considerati quattro criteri:
\begin{itemize}
\item Se le premesse su cui si regge l'argomentazione sono effettivamente vere.
\item Se la conclusione \`e probabile data la verit\`a delle premesse.
\item Se le premesse sono pertinenti alla conclusione.
\item Se la conclusione risulta vulnerabile di fronte a nuove informazioni.
\end{itemize}
\subsection{Verit\`a delle premesse}
Un'argomentazione non stabilisce la verit\`a della conclusione se una delle premesse \`e falsa. Inoltre se una delle premesse ha un grado di verit\`a ignoto allora si fallisce nel determinare il valore della 
conclusione per quanto ci \`e dato conoscere. 
\subsection{Probabilit\`a della conclusione}
Si valutano le argomentazioni in relazione alla probabilit\`a della conclusione data la verit\`a delle premesse. Le argomentazioni si classificano pertanto in deduttive e induttiva. Si dice deduttiva 
un'argomentazione la cui conclusione segue necessariamente dalle premesse di partenza mentre se esiste una certa probabilit\`a che la conclusione sia falsa l'argomentazione si dice induttiva. Si definisce 
pertanto la probabilit\`a induttiva, ovvero la probabilit\`a che date premesse tutte vere lo sia anche la conclusione. La forza induttiva di un'argomentazione pu\`o variare a seconda del contesto. 
Considerando argomentazioni complesse si noti che la validit\`a deduttiva e probabilit\`a induttiva dipendono dalle premesse fondamentali e la conclusione. Per valutarla si consideri:
\begin{itemize}
\item Nel caso di argomentazioni complesse non convergenti con almeno un passo debole la probabilit\`a induttiva risulta bassa.
\item Nel caso di argomentazioni complesse non convergenti con solo e non troppi passi forti la probabilit\`a induttiva risulta alta.
\item Nel caso di argomentazione convergente la probabilit\`a \`e tanto alta quanto la probabilit\`a del ramo pi\`u forte.
\item Se tutti i passi di un'argomentazione sono deduttivi allora \`e deduttiva anche l'argomentazione nella sua interezza.
\end{itemize}
\subsection{Pertinenza}
Un'argomentazione che manca di pertinenza non \`e utile per dimostrare la verit\`a della conclusione e si dice che commette una fallacia di pertinenza o rilevanza. La mancanza di pertinenza causa una 
discontinuit\`a tra premesse e conclusione. Per esempio una conclusione logicamente necessaria qualsiasi insieme di premesse la conclude. Un altro modo in cui un'argomentazione pu\`o mancare di pertinenza 
\`e quando ha premesse inconsistenti che non possono essere tutte simultaneamente vere. 
\subsection{Vulnerabilit\`a}
Le argomentazioni induttive soffrono di vulnerabilit\`a a fronte di una nuova premessa in quanto possono indebolire la probabilit\`a induttiva. Per questo motivo nel ragionamento induttivo la scelta delle 
premesse \`e fondamentale. Nasce perci\`o il criterio dell'evidenza totale che dice che se un'argomentazione \`e induttiva le sue premesse devono contenere tutta l'evidenza conosciuta che sia pertinente per la 
conclusione. 
\section{Fallacie di ragionamento}
Si dicono fallacie gli errori che danneggiano la cogenza delle argomentazioni in cui insorgono. Si divideranno e fallacie in cinque gruppi:
\begin{itemize}
\item Fallacie semantiche: derivano dall'uso di linguaggio ambiguo, ovvero espressioni il cui significato non \`e determinato in maniera chiara.
\item Fallacie formali: vengono commesse quando si applica una regola errata o una regola valida in modo errato.
\item Fallacie induttive: la probabilit\`a della soluzione \`e inferiore di quanto si suppone.
\item Fallacie di presunzione: derivano da ragionamenti in cui si presume la verit\`a di ci\`o di cui si vuole dimostrare.
\item Fallacie di pertinenza: derivano da ragionamenti le cui premesse non hanno relazione o scarsa relazione con la conclusione.
\end{itemize}