\chapter{Logica proposizionale}
\section{Simbolizzare \`e chiarificare}
Nella logica esistono varie notazioni simboliche artificiali con cui esprimere la struttura logica dei ragionamenti in modo da poterne verificare le condizioni di verit\`a. Essendo che la correttezza dei ragionamenti 
dipende dalla struttura, dai rapporti tra gli enunciati che li compongono e dal significato delle parole logiche la valutazione dei ragionamenti \`e resa difficoltosa dall'utilizzo del linguaggio naturale in quanti 
impreciso e metaforico. Viene pertanto teorizzato un simbolismo logico perfetto che rimedia ai difetti delle lingue naturali e che permette di descrivere la scienza rigorosa. 
\section{Determinatezza, bivalenza, vero-funzionalit\`a}
Per descrivere un linguaggio logico \`e necessario considerare tre elementi:
\begin{itemize}
\item Principio di determinatezza: ogni enunciato ha uno e un solo valore di verit\`a.
\item Principio di bivalenza: i valori di verit\`a sono soltanto due: vero o valso. Combinato con il primo ci dice che ogni enunciato ha uno e uno solo
dei due valori di verit\`a. 
\item Vero funzionalit\`a o estensibilit\`a: lo stato o valore di verit\`a degli enunciati composti dipende interamente da quello degli enunciati che li
compongono. In altro modo, il valore di verit\`a di enunciati dipende o \`e funzione del valore di verit\`a degli enunciati che li compongono. 
\end{itemize}
Il principio di vero funzionalit\`a non vale in tutti i contesti: \`e possibile che per alcuni enunciati sostituire un enunciato elementare con un altro dallo stesso valore di verit\`a cambi lo stato di verit\`a 
dell'enunciato composto. Questi casi di fallimento della sostituzione danno luogo a elementi non vero-funzionali o non estensionali prodotti tipicamente da espressioni modali o epistemiche.
\section{Connettivi logici}
I connettivi logici o connettivi enunciativi sono quelle espressioni che consentono di formare enunciati composti connettendo tra di loro enunciati dati. Si possono chiamare anche connettivi vero-funzionali. In 
questo capitolo verr\`a analizzato il linguaggio enunciativo o proposizionale con cui ci si propone di evidenziare la forma logica di ragionamenti la cui correttezza dipende dalla struttura connettivale degli 
enunciati che li compongono. Gli enunciati semplici vengono sostituiti da singole lettere dette variabili enunciative o proposizionali che possono sostituire un qualunque enunciato semplice. Per
i simboli logici vengono utilizzati simboli appositi.
\subsection{Congiunzione}
Il connettivo logico della congiunzione traduce la "e". \`E un connettivo binario e si indica con il simbolo $\land$. I due enunciati connetti verranno detti congiunti. Questo connettivo \`e un connettivo vero-
funzionale e il suo valore di verit\`a dipende unicamente da quello dei suoi componenti e in particolare risulter\`a vero se e solo se entrambi i suoi congiunti sono veri. Il valore di verit\`a pu\`o pertanto essere 
rappresentato da una tabella:
\begin{tabular}{|c|c|c|}
\hline
$P$ & $Q$ & $P\land Q$\\
\hline
T & T & T \\
\hline
T & F & F \\
\hline
F & T & F \\
\hline
F & F & F \\
\hline
\end{tabular}
Questa tabella si dice matrice del connettivo e pu\`o essere intesa come una sua definizione in quanto esprime la funzione di verit\`a che il connettivo esprime. 
\subsection{Disgiunzione}
Il connettivo logico della disgiunzione traduce la "o". \`E un connettivo binario e si indica con il simbolo $\lor$. I due enunciati verranno detti disgiunti. Questo connettivo ha il significato della disgiunzione 
debole o inclusiva e risulta falsa solo quando entrambi i disgiunti sono falsi:
\begin{tabular}{|c|c|c|}
\hline
$P$ & $Q$ & $P\lor Q$\\
\hline
T & T & T \\
\hline
T & F & T \\
\hline
F & T & T \\
\hline
F & F & F \\
\hline
\end{tabular}
\subsection{Negazione}
Il connettivo logico della negazione traduce il "non". Viene rappresentato con il simbolo $\neg$ o $\sim$. \`E un connettivo unario in quanto si applica ad un unico enunciato per volta. 
\begin{tabular}{|c|c|}
\hline
$P$ & $\neg P$ \\
\hline
T & F \\
\hline
F & T \\
\hline
\end{tabular}
Questo connettivo ha il ruolo di invertire il valore di verit\`a dell'enunciato. Si nota come $\neg\neg P\equiv P$. Si dice che due enunciati sono contradditori se e solo se uno deve essere vero se l'altro \`e falso e 
viceversa. 
\subsection{Condizione materiale}
Il connettivo binario del condizionale intende tradurre l'espressione "se...allora...". Si nota con il simbolo $\Rightarrow$. Il primo enunciato \`e detto antecedente, il secondo conseguente. Si intende che un 
enunciato condizionale \`e falso nel caso in cui il suo antecedente sia vero e il conseguente falso, ovvero si intende escludere che all'antecedente della sua assezione accada di essere vero mentre il conseguente 
\`e falso.
\begin{tabular}{|c|c|c|}
\hline
$P$ & $Q$ & $P\Rightarrow Q$\\
\hline
T & T & T \\
\hline
T & F & F \\
\hline
F & T & T \\
\hline
F & F & T \\
\hline
\end{tabular}
Viene anche chiamato condizionale materiale. Questo connettivo esprime la nozione di condizione sufficiente: la verit\`a del suo antecedente \`e condizione sufficiente per la verit\`a del suo conseguente in 
quanto la permette sia al suo verificarsi che attraverso altre condizioni. 
\subsection{Bicondizionale}
Questo connettivo logico intende tradurre l'espressione "se e solo se". Viene utilizzato il simbolo $\Leftrightarrow$. Gli enunciati costituenti sono detti lato sinistro e destro e il composto risulta vero solo se i 
due sono entrambi con lo stesso valore di verit\`a:
\begin{tabular}{|c|c|c|}
\hline
$P$ & $Q$ & $P\Leftrightarrow Q$\\
\hline
T & T & T \\
\hline
T & F & F \\
\hline
F & T & F \\
\hline
F & F & T \\
\hline
\end{tabular}
\`E anche detto equivalenza materiale e si dice che due enunciati sono materialmente equivalenti quando hanno lo stesso valore di verit\`a. Equivale alla congiunzione di due condizionali in cui l'antecedente 
dell'uno \`e il conseguente dell'altro. Si pu\`o pertanto dire che: $P\Leftrightarrow Q:=(P\Rightarrow Q)\land (Q\Rightarrow P)$. Questo connettivo permette di esprimere il concetto di condizione necessaria e 
sufficiente.
\section{Regole per mettere in campo una buona formazione}
\subsubsection{Parentesi}
In maniera del tutto analoga alle formule matematiche servono per evitare ambiguit\`a e dare un ordine di utilizzo di considerazione ai connettivi logici.
\subsection{Simboli}
Il linguaggio logico possiede un alfabeto logico formato dai cinque connettivi logici e un alfabeto descrittivo consistente in un numero indefinito di variabili enunciative che pu\`o essere pensato come 
indefinitamente esteso. \`E importante distinguere i due alfabeti: quello descrittivo si riferisce ad enunciati che descrivono il mondo e parlano di stati di cose, quello logico possiede elementi non descrittivi che 
hanno unicamente la funzione di connettivi e operatori. Esister\`a inoltre un alfabeto ausiliario che in questo caso consiste unicamente delle parentesi.
\subsection{Formule ben formate}
Un linguaggio logico-formale si considera dato quando \`e stato specificato insieme al suo alfabeto anche il modo in cui tali simboli possono essere combinati per formare espressioni composte. Viene definita 
come formula del linguaggio una qualunque sequenza finita di simboli. La costruzione di un linguaggio formale esige precise regole morfologiche o di formazione che ne definiscono la sintassi che specificano 
un sottoinsieme di formule ammissibili dette sintatticamente ben formate. Si noti come esistono un numero infinito di enunciati e formule mentre si devono specificare le regole di formazione in 
maniera finita e assolutamente rigorosa. Le regole devono rendere possibile, di fronte ad una qualunque formula di decidere se \`e ben formata o meno. Si rende pertanto necessario procedere per induzione. 
\subsection{Come catturare un infinito in modo finito}
Per definire un insieme finito si rende necessario l'utilizzo di tre clausole:
\begin{itemize}
\item Base dell'induzione: si specifica un certo gruppo iniziale di elementi appartenenti all'insieme.
\item Passo dell'induzione: si specificano delle operazioni e si dice che applicando tali operazioni a elementi dell'insieme si ottengono elementi 
dell'insieme.
\item Conclusione o chiusura dell'induzione: si specifica che appartengono unicamente all'insieme unicamente quelli che soddisfano i primi due passi.
\end{itemize}
In questo modo viene rigorosamente specificato un insieme con cardinalit\`a infinita.
\subsection{Metavariabili e regole di formazione}
Le regole di formazione di un linguaggio formale consistono in una serie di clausole che forniscono una definizione induttiva di formula ben formata, identificando tutte e sole le  formule ben formate. A questo 
scopo si deve introdurre la nozione di metavariabile o variabili metalogiche o metalinguistiche sono simboli che rappresentano simboli o sequenze di simboli del linguaggio formale. Verranno ora introdotte 
formule per queste metavariabili, ovvero valide per singole variabili enunciative o per sequenze di esse. Essendo le formule in numero indefinito, sar\`a necessario avere un numero indefinito di 
metavariabili, che verranno identificate con le lettere greche. 
\begin{itemize}
\item Base induttiva: ogni variabile enunciativa \`e una formula ben formata.
\item Passo induttivo, considerando $\alpha$ e $\beta$ formule ben formate:
\begin{itemize}
\item $\neg\alpha$ \`e una formula ben formata.
\item $\alpha\land\beta$ \`e una formula ben formata.
\item $\alpha\lor\beta$ \`e una formula ben formata.
\item $\alpha\Rightarrow\beta$ \`e una formula ben formata.
\item $\alpha\Leftrightarrow\beta$ \`e una formula ben formata.
\end{itemize}
\item Chiusura dell'induzione: le uniche formule ben formate sono quelle che si possono ricavare dalla base e dal passo induttivo.
\end{itemize}
Questo sta a significare che sono formule ben formate tutte le variabili enunciative e tutti i composti che si ottengono connettendo formule ben formate date in modo conforme alle regole mediante i cinque 
connettivi logici con gli eventuali simboli ausiliari. 
\subsection{Campo, subordinazione di connettivi}
Essendo le parentesi meri ausili alla comprensione della formula verranno omesse quando questo passo non crea ambiguit\`a, pertanto possono essere omesse agli estremi di una formula, e attraverso 
l'occorrenza di un simbolo, il campo e la subordinazione di un connettivo. Si indica per occorrenza di un simbolo in una formula il numero di volte in cui compare, ordinandolo da sinistra  a destra. Il campo o 
ambito dell'occorrenza di un connettivo in una formula \`e la pi\`u piccola formula ben formata in cui tale occorrenza compare. Le parentesi hanno lo scopo di chiarire l'ambito di ciascun connettivo. Grazie alla 
nozione di campo si pu\`o definire quella di subordinazione di un connettivo: si dice subordinato a un altro se e solo se il suo campo \`e contenuto nel campo di quest ultimo. Si dice pertanto che la formula 
subordinata \`e una sottoformula e connettivo principale il connettivo non subordinato a nessun altro. Si stabilisce anche un ordine di forza dei connettivi in questo ordine:$\neg$,  $\lor$ e $\land$ e $
\Leftrightarrow, \Rightarrow$ e stabilendo che un connettivo che lega pi\`u fortemente attrae nel proprio campo una certa formula o variabile enunciativa. Congiunzioni e disgiunzioni sono associative. 
\section{Creare tavole di verit\`a}
\subsection{Valutare i ragionamente con le tavole di verit\`a}
Il metodo delle tavole di verit\`a pu\`o essere utilizzato come un sistema di calcolo logico, in modo da stabilire in modo meccanico la correttezza o la non correttezza di tutti gli argomenti la cui forma logica sia 
esprimibile nel linguaggio enunciativo. Le premesse sono intervallate da virgole e sono separate dalla conclusione dal segno $/$. Questo schema cattura la forma logica di una quantit\`a di elementi espressi in 
linguaggio naturale. Si pu\`o dare pertanto una prima caratterizzazione di ragionamento espresso nel linguaggio formale: un ragionamento \`e una sequenza finita di formule dette premesse intervallate da 
virgole seguite dal segno $/$ che precede la conclusione, una formula. Ricordando il criterio di correttezza logica un argomento \`e corretto se e solo se non pu\`o darsi che tutte le sue premesse siano vere e la 
conclusione falsa. Mediante una tavola di verit\`a diventa pertanto possibile assegnare il valore vero o falso a ciascuna delle variabili enunciative che compongono le premesse e la conclusione degli schemi 
d'argomento. Eseguendo quest'assegnazione per ogni combinazione di valori di verit\`a si verificher\`a che se vi \`e una sola assegnazione in cui tutte le premesse risultano vere ma la conclusione falsa lo schema 
\`e scorretto, se tutti i casi in cui le assegnazioni di valori di verit\`a alle variabili enunciative rendono vere tutte le premesse e la conclusione lo schema \`e corretto. Gli argomenti scorretti vengono chiamati 
fallacie logiche. 
\subsection{Tautologie, incoerenze e contingenze}
Le leggi logiche pi\`u note sono il principio di identit\`a, di non contraddizione e del terzo escluso.  
\subsubsection{Principio di identit\`a}
$\alpha\Rightarrow \alpha$ ovvero ogni enunciato implica s\`e stesso, viene pertanto detta anche legge di identit\`a enunciativa. 
\subsubsection{Principio di non contraddizione}
$\neg(\alpha\land\neg \alpha)$ ovvero ogni enunciato non si d\`a il caso che valgano l'enunciato stesso e la sua negazione.
\subsubsection{Principio del terzo escluso}
$\alpha\lor\neg \alpha$ ovvero per qualsiasi enunciato valgono l'enunciato stesso o la sua negazione.
\subsubsection{Tautologie}
Si dice tautologia una qualsiasi formula che assume il valore di vero per qualunque valore di verit\`a assegnato alle sue variabili enunciative. Vengono dette anche leggi logico-enunciative e si nota perch\`e 
prendono il nome di leggi. Sono infatti vere a prescindere del valore di verit\`a assegnato alle variabili enunciative in virt\`u della loro forma logica. 
\subsubsection{Incoerenze o contraddizioni}
Si dice incoerenza o contraddizione una qualsiasi formula che assume il valore di falso per qualunque valore di verit\`a assegnato alle sue variabili enunciative. La negazione di una tautologia \`e un'incoerenza e 
viceversa.
\subsubsection{Contingenza}
Si dice contingenza una formula il cui valore di verit\`a dipende dai valori di verit\`a delle sue variabili enunciative. 
\subsubsection{Decidibilit\`a}
Le tabelle di verit\`a permettono di classificare univocamente le formule in tautologie, contraddizioni o contingenze o la correttezza del ragionamento. Essendo questo fattibile in un numero finito di passi si dice 
che la questione di stabilire le leggi logico-enunciative \`e decidibile. Il limite delle tabelle di verit\`a \`e la loro complessit\`a: rendendosi necessario sviluppare tutte le possibili combinazioni per le variabili 
enunciative un algoritmo per la decisione avr\`a complessit\`a $O(2^n)$, dove $n$ \`e il numero di variabili enunciative. 
\subsection{La foma condizionale corrispondente}
Dato un qualunque schema di ragionamento nella forma generale $\alpha_1,\dots, \alpha_n/\beta$ si definisce la sua forma condizionale corrispondente come la formula: $\alpha_1\land\dots\land 
\alpha_n\Rightarrow\beta$. Questa forma \`e quel condizionale che ha per antecedente la congiunzione delle premesse dell'argomento e come conseguente la conclusione. Per qualunque schema vale che \`e 
corretto se e solo se la sua forma condizionale corrispondente \`e una tautologia. 
\section{Alberi di refutazione}
Creati per snellire il lavoro delle tavole di verit\`a, gli alberi di refutazione sono una ricerca esaustiva dei modi in cui tutte le formule di una lista possano essere vere. Si comincia formando una lista composta 
dalle sue premesse e dalla negazione della sua conclusione e si scompone ogni formula segnandola fino ad ottenere lettere enunciative o loro negazioni. Se si trova qualche assegnazione di un valore di verit\`a 
alle lettere enunciative che rende vere tutte le formule della lista allora risulta che la forma \`e stata refutata e si pu\`o sancirne l'invalidit\`a, se invece la ricerca non permette di scoprire alcuna assegnazione di 
valori di verit\`a che lo permetta allora la forma \`e valida. Se alla fine di un ramo dell'albero il tentativo di refutazione fallisce lo si segna con una $X$ e si chiama cammino chiuso, altrimenti aperto. Un cammino 
si dice terminato se \`e chiuso o se le sole formule non segnate che contiene sono elementari o loro negazioni. Se tutti i cammini dell'albero terminato sono chiusi la forma \`e valida, mentre se ne rimane uno 
aperto \`e refutata. In particolare ogni cammino aperto dell'albero terminato rappresenta un modello per la costruzione di un controesempio. 
\subsection{Regole sintattiche}
\begin{itemize}
\item Le regole per la costruzione degli alberi si applicano solo a formule intere e non a sottoformule.
\item L'ordine di applicazione non \`e importante ma \`e pi\`u efficace utilizzare prima regole che non danno luogo a ramificazioni. 
\item In un albero terminato per una forma argomentativa i cammini aperti indicano tutti i controesempi a quella formula. 
\end{itemize}
\subsection{Regole di formazione}
\begin{itemize}
\item Negazione: se un cammino aperto contiene sia una formula che la sia negazione si scriva una $X$ al termine del cammino.
\item Congiunzione: se un cammino aperto contiene un formula non segnata nella formula $\alpha\land\beta$ la si segni e si scriva sia $\alpha$ che $\beta$  al termine di ogni cammino aperto che la contiene.
\item Disgiunzione: se un cammino aperto contiene una disgiunzione non segnata si traccino due rami sotto ciascun cammino aperto che la contiene e si scriva un disgiunto per ramo.
\item Condizionale: se un cammino contiene una condizionale non segnata la si segni e si traccino due rami sotto ciascun cammino aperto che la contiene e si scriva la negazione dell'antecedente sul primo e il 
conseguente sul secondo. 
\item Bicondizionale: se un cammino contiene una bicondizionale non segnata la si segni e si traccino due rami sotto ciascun cammino aperto che la contiene e si scriva lato sinistro e destro sul primo e la 
negazione dei due sul secondo.
\item Negazione negata: se un cammino contiene una doppia negazione non segnata la si segni e si scriva la formula su ogni ramo che la contiene. 
\item Congiunzione negata: se un cammino aperto contiene un formula non segnata nella formula $\neg(\alpha\land\beta)$ la si segni, si traccino due cammini  e si scriva $\neg\alpha$ sul primo e $\neg\beta$ 
al termine del secondo.
\item Disgiunzione negata: se un cammino aperto contiene un formula non segnata nella forma $\neg(\alpha\lor\beta)$ la si segni e si scriva sia $\neg\alpha$ che $\neg\beta$  al termine di ogni cammino 
aperto che la contiene.
\item Congiunzione negata: se un cammino aperto contiene una formula non segnata nella forma $\neg(\alpha\Rightarrow\beta)$ la si segni e si scriva sia $\alpha$ che $\neg\beta$ al termine di ogni cammino 
aperto che la contiene. 
\item Bicondizionale negata: se un cammino aperto contiene una formula non segnata nella forma $\neg(\alpha\Leftrightarrow\beta)$ la si segni, si traccino due rami sotto ciascun cammino aperto che la 
contiene e scrivere $\alpha$ e $\neg\beta$ sotto il primo e $\neg\alpha$ e $\beta$ sotto il secondo. 
\end{itemize}
\subsection{Altri utilizzi per gli alberi di refutazione}
Inoltre un albero terminato con un cammino aperto costituisce  un insieme vero-funzionalmente consistente. In particolare:
\begin{itemize}
\item $\alpha$ \`e inconsistente se e solo se tutti i cammini di un albero terminato per $\alpha$ sono chiusi.
\item $\alpha$ \`e tautologica se e solo se tutti i cammini di un albero terminato per $\neg\alpha$ sono chiusi.
\item $\alpha$ \`e vero-funzionalmente contingente se e solo se $\alpha$ non \`e n\`e tautologica n\`e inconsistente.
\end{itemize}