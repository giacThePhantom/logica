\chapter{Cenni di metalogica}
Una volta determinato un linguaggio logico $L$ e a un sistema formale $S$ su di esso ci si pu\`o occupare delle caratteristiche generali del sistema entrando
cos\`i nel campo della metalogica. Si tratta di un discorso che verte sulle caratteristiche complessive dei sistemi logici. I teoremi della metalogica qui 
discussi vengono chiamati metateoremi e vertono sui sistemi formali esprimendo le loro propriet\`a generali.
\section{Coerenza e completezza o le virt\`u di una logica}
\subsection{La coerenza dei sistemi formali}
Un sistema formale $S$ viene detto sintatticamente coerente se per nessuna formula $\alpha$ del linguaggio formale su cui \`e impiantato si d\`a il caso che 
$\vdash\alpha$ e $\vdash\neg\alpha$ ossia se non consente di derivare come teoremi sua una formila che la sua negazione. Se ci\`o accade il sistema\`e detto
incoerente o contradditorio. Per capire le conseguenze di un sistema incoerente si introduca l'inconsistenza: un sistema $S$ viene detto inconsistente se 
consente di dimostrare o derivare come teoremi tutte le formule del linguaggio $L$ su cui \`e impiantato. Se ne esiste una che $S$ non dimostra viene detto 
consistente. In un sistema formale che esprime la logica classica e nei sistemi dotati di $E\neg$ o di formulazioni equivalenti della legge dello pseudo-
Scoto l'incoerenza implica l'inconsistenza: questa regola dice che da una contraddizione si pu\`o dedurre qualunque formula. Pertanto se fosse possiblie 
dedurre una contraddizione come teorema il sistema in questione diventerebbe inconsistente che sarebbe deduttivamente inutile in quanto dimostra tutto. Da 
qui l'importanza di fornire dimostrazioni di coerenza o non contraddittoreit\`a che garantiscano che i principi di un sistema formale siano sicuri. La 
dimostrazione della coerenza di un sistema formale avviene utilizzando la coerenza o correttezza semantica in quanto caratterizzata con riferimento alla 
nozione di verit\`a. Si studi la correttezza del calcolo dei predicati classico. Di questo il sistema della deduzione naturale \`e una sistemazione tipica.
Provare che \`e corretto vuol dire provare che tutti i teoremi del calcolo dei predicati sono veri in ogni interpretazione, ovvero in tutti i modelli: se
$\vdash\alpha$ allora $\models\alpha$. Questo \`e detto il teorema speciale o debole di coerenza del calcolo dei predicati che riguarda il caso a zero 
premesse. Il teorema generale o forte di correttezza dice che tutte le formule derivabili da un gruppo qualsiasi di premesse nel calcolo dei predicati sono 
conseguenza logica di quel gruppo di premesse: se $\alpha_1, \dots, \alpha_n\vdash\beta$ allora $\alpha_1,\dots, \alpha_n\models\beta$, la dimostrazione non
vista nel dettaglio ha luogo per induzione sulla costruzione delle dimostrazioni formali. Il teorema debole dice che le versioni del calcolo dei predicati 
consentono di derivare come teoremi solo formule vere in ogni interpretazione, o le leggi logiche. Da questa correttezza semantica deriva la coerenza 
sintattica: non essendo la contraddizione una legge logica non potr\`a mai essere derivata: non \`e vera in nessuna interpretazione. Il teorema forte dice 
che i principi e regole d'inferenza delle versioni del calcolo dei predicati consentono di dedurre solo conseguenze logiche: essendo che la conseguenza
logica cattura il criterio di correttezza logica dei ragionamenti e il teorema assicura che il calcolo dei predicati consente di costruire dimostrazioni
formali per ragionamenti effettivamente validi non consentendo mai di derivare una conclusione falsa da premesse tutte vere.
\subsection{La completezza dei sistemi formali}
La completezza \`e la propriet\`a che dice che i sistemi "dimostrano il pi\`u possibile". Il teorema deboel di completezza afferma che tutte le formule vere 
in ogni interpretazione sono teoremi del calcolo dei predicati: se $\models\alpha$ allora $\vdash\alpha$ ovvero nel calcolo dei predicati si possono 
derivare come predicati tutte le leggi logiche. Unendo i teoremi deboli di coerenza e completezza si otterr\`a: $\vdash\alpha$ se e solo se $\models\alpha$ 
che dice come nel caso della logica dei predicati classica il teorema del calcolo logico e quello di legge logica sono nozioni indipendenti che individuano
lo stesso insieme di formule. Il teorema forte di completezza dice che tutte le conseguenze logiche di un gruppo qualsiasi di premesse sono derivabili da 
quel gruppo di premesse nel calcolo dei predicati: se $\alpha_1, \dots, \alpha_n\models\beta$ allora $\alpha_1,\dots, \alpha_n\vdash\beta$, ossia nel calcolo
dei predicati si possono costruire dimostrazoni formali per tutti i ragionamenti effettivamente validi. Queste due propriet\`a dicono come la logica classica
non sia arbitraria in quanto le inferenze dimostrabili sono corrette contemporaneamente sintatticamente e semanticamente e che i vari sistemi formali che
la esprimono sono equivalenti fra loro in quanto coerenti e completi. 
\section{Teoremi limitativi}
\subsection{L'incompletezza dell'aritmetica}
In un qualsiasi sistema formale $S$ che soddisfi certe condizioni si pu\`o parlare di alcune propriet\`a e relazioni sintattiche che riguardano $S$ stesso, 
nel senso che la propriet\`a sintattica di essere dimostrabile in $S$ pu\`o essere espressa all'interno della teoria stessa. Per ottenere ci\`o si rende 
necessaria la g\"odelizzazione: studiando un qualunque linguaggio formalizzato $L$ e un sistema $S$ si du esso impiantato si ha a che fare con un insieme 
numerabile di oggetti e che pertanto si possono rappresentare associandoli ai numeri naturali. La g\"odelizzazione \`e basata su questa associazione univoca
di simboli, formule e sequenze di formule di $L$ a un numero naturale. Successivamente data un'espressione di $L$ si pu\`o stabilire il numero di G\"odel
che le corrisponde e dato un numero naturale si pu\`o stabilire se e a quale espressione si riferisce. Se ora il sistema ha capacit\`a espressive 
riconducibili al fatto che contiene una certa quantit\`a di aritmetica formale \`e possibile che enunciati del sistema parlino di propriet\`a e ralzioni tra
enunciati del sistema stesso: i numeri natorali diventano gli oggetti di cui gli enunciati aritmetici parlano ma codici univocamente associati a espressioni
del linguaggio della teoria stessa, allora affermazioni si $S$ sono rispecchiate in $S$ come affermazioni su numeri. Utilizzando la diagonalizzazione si 
possono costruire enunciati che parlano di s\`e stessi in quato si riferiscono al proprio numero di G\"odel. Si costruisca pertanto in $S$ $\gamma$ che
dice di essere indimostrabile in $S$: $\gamma\Leftrightarrow\neg Dim("\gamma")$. In luogo del predicato di verit\`a $V$ si ha un predicato di 
dimostrabilit\`a. La formula $Dim(x)$ \`e una fromula aritmetica che raffigura nel sistema la propriet\`a sintattica di formule del sistema di essere
dimostrabile in $S$, ovvero di essere un teorema in $S$. Questo predicato \`e esprimibile in $S$ e non d\`a luogo a paradossi che dice che per un sistema
$S$ che soddisfi queste condizioni se $S$ \`e coerente allora non $\vdash\gamma$ dove si intende con coerenza il fatto che $S$ dimostra solo verit\`a in 
quanto se $\gamma$ fosse dimostrabile e pertanto falso il sistema consentirebbe di derivare enunciati falsi come teoremi. Se $\gamma$ non \`e dimostrabile
allora $\gamma$ \`e vero, pertanto se $S$ \`e coerente allora $\vdash\neg\gamma$ in quanto se fosse vero la sua negazione sar\`a falsa ed essendo $S$ 
coerente $\neg\gamma$ non sar\`a teorema di $S$. Queste due condizioni unite costituiscono una versione semantica informale del primo teorema di 
incomletezza di G\"oedel. Il primo teorema dice che qualunque sistema formale coerente $S$ in grado di esprimere l'aritmetica elementare \`e incompleto:
si possono formulare enunciati aritmetici veri nel suo linguaggio ma $S$ non pu\`o derivarli come teoremi. Inoltre $\gamma$ \`e indecidibile per $S$ in 
quanto non pu\`o dimostarlo n\`e refutarlo in quanto non pu\`o dimostrare n\`e lui n\`e la sua negazione. E aggiungere questo problema indecidibile come
assioma ne creerebbe altri.
\subsection{Incompletezza e prove di coerenza}
Si \`e provato come si $S$ \`e coerente allora $\gamma$ non \`e dimostrabile. Per esprimere questa idea nel sistema formale bisogna introdurre 
un'affermazione di coerenza ($Coer$) per $S$, dove $Coer:=\neg Dim("k")$ ovvero dire che $S$ \`e coerente \`e come dire che $S$ non dimostra $k$ una 
qualsiasi falsit\`a logica o aritmetica. Si pu\`o esprimre che se $S$ \`e coerente allora $\gamma$ non \`e dimostrabile: $\neg Dim("k")\Rightarrow\neg 
Dim("\gamma")$. Il coonseguente di quest'ultima \`e $\gamma$, pertanto $Coer\Rightarrow\gamma$, pertanto se $S$ \`e coerente allora non $\vdash Coer$
Supponendo di poter dimostare l'antecedente che la formula esprimere la coerenza di $S$ sia un teorema del sistema di esso mediante il modus ponens si 
potrebbe dimostrare anche $\gamma$, quello esclusio dal primo teorema di incompletezza, viene pertanto espresso il secondo teorema di incompletezza di 
G\"oedel che \`e un corollario del primo che dice che se $S$ \`e coerente allora non \`e in grado di dimostrare l'assezione $Coerr$ del linguaggio formale
su cui \`e impiantato che esprima $S$, ovvero $S$ non \`e in grado di dimostrare la propria coerenza. 
\subsection{La logica \`e trasendentale}
Questi due teoremi si ritiene illustrino una discrepanza tra dimostrabilit\`a in un sistema formale e verit\`a in quanto nessun sistema formale pu\`o 
dimostrare solo cose vere ma non tutte e nessun sistema coerente \`e capace di dimostrare autonomamente di esserlo. 