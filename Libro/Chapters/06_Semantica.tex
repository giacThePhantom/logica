\chapter{Semantica logica}
In generale si ascrive alla semantica tutti ci\`o che riguarda il rapporto fra i segni linguistici e le entit\`a che essi possono significare. \`E strettamente connessa all'ontologia, lo studio dei complessi di enti che si 
possono esprimere nel linguaggio formale. Verranno trattati i tratti essenziali della semantica logica elementare standard. Per parlare di ontologia si rendono necessari richiami alla teoria degli insiemi.
\section{Teoria degli insiemi}
\subsection{Insiemi}
Si definisce un insieme come una qualsiasi collezione di oggetti rappresentati da lettere maiuscole i cui elementi sono indicate con lettere minuscole. La relazione fondamentale \`e l'appartenenza $\in$ che 
determina che un oggetto appartiene ad un insieme. Un insieme pu\`o essere identificato attraverso una lista completa dei suoi elementi tra parentesi classe e non contano n\`e l'ordine n\`e le ripetizioni. Per 
rappresentare insiemi infiniti viene utilizzata una condizione di appartenenza che indica la caratteristica comune posseduta da tutti e soli gli elementi dell'insieme in questione. In generale una qualsiasi 
propriet\`a $F$ determina un'insieme (principio di astrazione). Il principio di estensionalit\`a stabilisce le condizioni sufficienti per l'identit\`a tra insiemi: $\forall x(x\in A\Leftrightarrow x\in B)\Rightarrow 
A=B$. Un insieme \`e determinato interamente dalla totalit\`a dei suoi elementi, in particolare tutti gli insiemi vuoti sono identici tra di loro, ovvero $\forall x(x\not\in A\land x\not\in B)\Rightarrow\forall x(x\in A 
\Leftrightarrow x\in B)\Rightarrow A=B$. Pertanto si identifica l'unico insieme vuoto con $\emptyset=\{x|x\neq x\}$. All'estremo opposto dell'insieme vuoto si trova l'insieme universo di cui ogni oggetto \`e 
membro indicato con $V=\{x|x=x\}$.
\subsection{Relazioni e operazioni insiemistiche}
La prima relazione da considerare \`e quella di inclusione o $\subseteq$ che indica la propriet\`a di un sottoinsieme, ovvero: $A\subseteq B\Leftrightarrow\forall x(x\in A\Rightarrow x\in B)$ a differenza della 
relazione di appartenenza che rapporta oggetti e insiemi questa mette in relazione insiemi con insiemi. Si deve pertanto distinguere tra l'oggetto $x$ e il singoletto $[x]$, ovvero l'insieme che ha come unico 
oggetto $x$. Questa relazione \`e riflessiva e transitiva. L'insieme vuoto \`e incluso in ogni insieme. 
\subsubsection{Operazioni tra insiemi}
\begin{itemize}
\item Unione: $A\cup B:=\{x|x\in A\lor x\in B\}$.
\item Intersezione: $A\cap B:=\{x|x\in A\land x\in B\}$.
\item Complementazione: $A':=\{x|x\not\in A\}$.
\end{itemize}
\subsubsection{Prodotto cartesiano}
Si definisce ennupla ordinata un insieme di $n$ elementi in cui l'ordine \`e specifico e serve a determinarla, si indica con $<x_1, \dots, x_n>$. Si dice prodotto cartesiano $A\times B$ un insieme formato da 
tutte e sole le coppie ordinate il cui primo elemento appartiene al primo insieme e il secondo al secondo: $A_1\times\dots\times A_n:=\{<x_1,\dots,  x_n>|x_1\in A_1\land\dots\land x_n\in A_n\}$. Si definisce 
potenza cartesiana il prodotto di un insieme con s\`e stesso.
\subsubsection{Insieme potenza}
Si definisce insieme potenza l'insieme formato da tutti i sottoinsiemi di un insieme: $P(A):=\{B|B\subseteq B\}$. 
\subsection{Ontologia e insiemi}
Attraverso la teoria degli insiemi si possono esprimere le strutture ontologiche che si assumono come sistemi di significati da attribuire alle espressioni del linguaggio formale. L'ontologia logica elementare si 
basa infatti su concetti che sono esprimibili mediante la teoria degli insiemi. Quando si interpretano espressioni dei linguaggi formali della logica si attribuisce loro un riferimento in strutture ontologiche dette 
universi del discorso, i mondi dove si pu\`o far parlare i linguaggi formali. Una struttura \`e normalmente formata da un insieme non vuoto $U$ di individui detto dominio o supporto alla struttura i quali godono 
di certe propriet\`a e fra cui sussistono relazioni e operazioni definite in $U$. In questo modo si possono esprimere i concetti di propriet\`a, relazione, operazione o funzione mediante insiemi in modo naturale. 
Assegnare una propriet\`a a certi individui vuol dire creare un sottoinsieme di $U$ detto estensione ontologica. Una relazione ennaria \`e un sottoinsieme dell'ennesima potenza cartesiana di $U$.
\subsection{Paradosso di Russel}
Esiste la possibilit\`a che esistano degli insiemi che contengano loro stessi. Gli insiemi che non appartengono a loro stessi si dicono normali e il loro insieme viene definito come $R=\{x|x\not\in x\}$. Questo 
insieme esiste in virt\`u di comprensione secondo cui qualsiasi condizione o propriet\`a determina un insieme. Ora ci si pu\`o chiedere se $R$ appartenga a s\`e stesso: da questo si ottiene che $R\in 
R\Leftrightarrow R\not\in R$ da cui per la logica elementare discende una contraddizione: $R\in R\land R\not\in R$. Questo paradosso mostra come se il principio di comprensione viene assunto senza 
restrizioni porta a possibili contraddizioni. La soluzione sta nella teoria di tipi logici. Il meccanismo consiste nello sviluppare una gerarchia di tipi di oggetti: individui, insiemi, insiemi di insiemi e cos\`i via e 
determinare che ci\`o che fa parte di un tipo logico pu\`o essere membro solo di qualcosa che faccia parte del livello superiore. Come risultato un insieme pu\`o essere composto solo di oggetti omogenei, ovvero 
appartenenti allo stesso livello logico. Altre proposte si basano sul principio della limitazione di grandezza in cui si differisce tra insieme e classe: alcune estensioni di predicati o classi non possono essere 
considerate insiemi o oggetti dei quali chiedersi se appartengano ad altri insiemi o classi. 
\section{Una teoria del significato basata sulla verit\`a}
Comprendere un enunciato vuol dire coglierne il significato che consiste nelle sue condizioni di verit\`a. Un enunciato si presenta come la descrizione di un pezzo di realt\`a e il suo significato \`e noto quando si 
conoscono le condizioni in cui la descrizione che esso fornisce \`e adeguata, ovvero come deve essere fatto il mondo affinch\`e il significato sia vero. La concezione di significato cos\`i proposta \`e pertanto 
incentrata sulla verit\`a. Quest'ultima si colloca fra il livello segnico e simbolico del linguaggio e il livello ontologico. La semantica in questa istanza ha lo scopo di di determinare le condizioni sotto le quali un 
enunciato costituisce un'affermazione vera intorno all'universo del discorso in cui lo si interpreta. Viene pertanto detta semantica vero-condizionale. Questa idea di significato si lega al fatto che l'enunciato 
dichiarativo \`e l'unit\`a fondamentale della logica in quanto configurazione linguistica secondo cui si pu\`o parlare di condizioni di verit\`a e che il significato delle espressioni subenunciative consiste nel modo in 
cui esse contribuiscono al significato degli enunciati in cui compaiono. In questo modo si pu\`o interpretare come soltanto negli enunciati le parole hanno senso. Conoscere le condizioni di verit\`a di un enunciato 
non equivale a sapere che sia vero o falso in quanto si pu\`o comprendere il significato di un enunciato il cui valore di verit\`a risulta ignoto o pu\`o cambiare nel tempo. Si deve considerare inoltre il principio di 
composizionalit\`a del significato secondo cui il significato di un'espressione linguistica composta dipende funzionalmente dai significati dei sui costituenti. Una teoria semantica deve tener conto di come si 
comprendono potenzialmente infinite espressioni linguistiche nuove se sono sintatticamente ben formate. Questo fatto pu\`o essere spigato come la capacit\`a da parte di un parlante competente di effettuare 
un calcolo del valore semantico di ogni espressione composta partendo da un numero finito di costituenti gi\`a noti. La composizionalit\`a del significato opera anche a livello di enunciati composti: un enunciato 
composto vero-funzionalmente ha un significato che dipende dal significato degli enunciati che lo compongono e dalla loro composizione. Avendo detto che il significato degli enunciati consiste nelle 
circostanze che li rendono veri il valore semantico di ogni enunciato composto dipende composizionalmente dal valore semantico degli enunciati che lo compongono. 
\section{Semantica Tarskiana}
\subsection{La convenzione V}
Vengono forniti gli strumenti per edificare una semantica rigorosa per un ampio gruppo di linguaggi formali. Per fare questo si caratterizza la nozione di verit\`a relativamente ad un certo linguaggio 
accompagnando ad essa un metodo per esplicitare le condizioni di verit\`a degli enunciati di singoli linguaggi logici formalizzati. La verit\`a \`e considerata come una propriet\`a di enunciati: il predicato "\`e vero" 
si applica al nome dell'enunciato la cui verit\`a \`e attribuita. Si definisce linguaggio oggetto il linguaggio di cui si parla o intorno al quale si forma una certa teoria o trattazione che sar\`a formata in un altro 
linguaggio detto metalinguaggio. Secondo questa convezione linguaggio oggetto e metalinguaggio devono essere distinti. Si pu\`o in questo modo formulare lo schema di verit\`a come $N$ \`e vero (nel 
linguaggio $L$) se e solo se $T$, dove $N$ \`e il nome dell'enunciato di $L$ cui si ascrive la verit\`a e $T$ \`e la sua traduzione nel metalinguaggio. Questo schema aiuta a ben definire la verit\`a per un 
linguaggio in quanto si individua una condizione formale di adeguatezza per la definizione consistente nel fatto che questa non deve consentire di dedurre contraddizioni e una condizione materiale di 
adeguatezza che si esprime formulando la convenzione $V$: la definizione sar\`a adeguata se si potr\`a dedurne logicamente tutte le esemplificazioni dello schema di verit\`a, ossia se per ognuno degli enunciati 
del linguaggio-oggetto $L$ si potr\`a derivare dalla definizione il corrispondente condizionale in quanto determinerebbero l'estensione del predicato di verit\`a per il linguaggio in questione. Emerge come 
questa concezione di verit\`a sia corrispondentista in base a cui un enunciato \`e vero se e solo se corrisponde ai fatti. Lo schema di verit\`a fornisce un criterio guida generale per specificare le condizioni di 
verit\`a degli enunciati senza far ricorso a troppe nozioni. Si studi ora come queste nozioni si sviluppano nel linguaggio studiato precedentemente che verr\`a chiamato $L$. La sua semantica verr\`a sviluppata 
utilizzando come metalinguaggio l'italiano informale con notazione di tipo insiemistico. A differenza dei linguaggio naturali che sono gi\`a interpretati, ovvero le espressioni sono originariamente provviste di 
significato le formule dei linguaggi formali sono pure sequenze di simboli costruite in base alle regole sintattiche. Si \`e pertanto interessati a stabilire le condizioni di verit\`a di $L$ quando alle espressioni di 
$L$ viene attribuito un significato in universi del discorso, ovvero fare in modo che il linguaggio parli di un certo mondo e poi di stabilire sotto quali condizioni una certa formula \`e vera in quell'universo. In 
questa prospettiva si parler\`a di verit\`a o falsit\`a in relazione all'universo in cui $L$ viene interpretato. 
\subsection{Interpretazioni e assegnazioni}
Si consideri una struttura ontologica che ha per dominio un insieme non vuoto $U$ di individui, un modello $\mathcal{M}$ per $L$ \`e una coppia ordinata $\mathcal{M}=<U, i>$ dove $i$ \`e una funzione di 
interpretazione, ossia una funzione che assegna significati a espressioni di $L$. Un'interpretazione di  $L$ \`e un'attribuzione di significato a ogni simbolo descrittivo costante di $L$ meiante $i$ che 
assegner\`a significati ai simboli in modo che per ogni nome proprio o costante individuale $k$ il suo significato secondo $i$ in $\mathcal{M}$ o $\mathcal{M}(k)$ sar\`a un individuo appartenente a $U$, per 
ogni costante funtoriale ennaria $f$ $\mathcal{M}(f)$ assegnatole da $i$ in $\mathcal{M}$ sar\`a una certa operazione ennaria definita su $U$ e per ogni costante predicativa ennaria $P$, $\mathcal{M}(P)$ 
assegnatole da $i$ in $\mathcal{M}$ sar\`a una propriet\`a, o una relazione ennaria definita su $U$. In questo modo i predicati monadici significheranno propriet\`a ricondotte a insiemi e alle loro estensioni 
ontologiche. I predicati poliadici di $L$ significheranno relazioni fra enti di $U$, ovvero insiemi di ennuple ordinate di elementi di $U$. Inoltre \`e stato fissato univocamente il significato delle costanti 
descrittive: ogni simbolo descrittivo costante ha uno e un solo significato in $\mathcal{M}$. Questa semantica \`e centrata su un apparato referenzialista: assume che il significato delle espressioni 
subenunciative di $L$ consista nel riferimento o denotazione di oggetti del dominio di $U$. Le variabili individuali a differenza delle costanti possono stare per individui o assumere valori diversi e si assume di 
solito che l'interpretazione si limiti a fissare per ogni variabile individuale di $x$ un campo di variazione, ovvero l'insieme dei suoi valori possibili. Tuttavia in termini di esplicitazione delle condizioni di verit\`a 
delle formule risulta utile assegnare un valore determinato o denotazione temporanea che servir\`a per poter valutare le formule aperte che contengono variabili libere. Si dice pertanto che una asseggazione 
$a$ relativa a un modello $\mathcal{M}$ \`e l'attribuzione a ogni variabile $x$ di $L$ di un valore determinato $a(x)$ nel suo campo di variazione $U$. Occorre distinguere tra assegnazione, ossia 
l'applicazione delle variabili a un oggetto determinato nel loro dominio e l'interpretazione delle costanti. $a$ differisce da $i$ perch\`e si suppone che il significato di una costante resti fissato univocamente in 
un'interpretazione mentre si possono avere diverse assegnazioni di valore alle variabili entro una stessa interpretazione. Combinando queste due operazioni si pu\`o avere una notazione deteminata per tutti i 
termini individuali del linguaggio e si scrive $\mathcal{M}^a(t)$ per indicare la denotazione di un termine individuale $t$ relativa all'interpretazione del modello e all'assegnazione $a$ e si definisce per 
induzione la costruzione dei termini individuali di $L$:
\begin{itemize}
\item Base idnuttiva:
\begin{itemize}
\item Se $t$ \`e una costante individuale allora $\mathcal{M}^a(t):=\mathcal{M}(t)$.
\item Se $s$ \`e una variabile individuale allora $\mathcal{M}^a(s):=a(s)$.
\end{itemize}
\item Passo induttivo: se $t_1, \dots, t_n$ sono termini individuali e $f$ \`e una qualsiasi costante funtoriale ennaria allora $\mathcal{M}^a(f(t_1, \dots, t_n)):=\mathcal{M}(f)(\mathcal{M}^a(t_1), \dots, 
\mathcal{M}^a(t_n))$
\end{itemize}
Se $t$ non contiene variabili le assegnazioni di valori alle variabili non fanno differenza per esso e il suo singificato \`e interamente fissato dalla sola interpretazione delle costanti.
\subsection{La definizione ricorsiva}
Dopo aver attribuito significati a tutte le parti descrittive di $L$ ha senso tentare di esplicitare per ogni formula di $L$ le condizioni per cui \`e vera in relazione all'interpretazione effettuata. Se si riuscir\`a a 
fornire una definizione che consenta in linea di principio di poter dedurre tutti i bicondizionali della forma di $V$ si rispetter\`a la convenzione $V$ e fornita una soddisfacente caratterizzazione della verit\`a per 
$L$. Per far ci\`o si necessita di considerare che i bicondizionali che esemplificano $V$ sono infiniti in quanto le regole di formazione consentono di costruire potenzialmente infinite formule e per ciascuna si 
rende necessario poter derivare il corrispondente bicondizionale. Si vuole pertanto poter esplicitare le condizioni di verit\`a di un'infinit\`a di formule in modo finito partendo da una definizione che contenga un 
numero finito di clausole. Per far questo si utilizzi l'induzione sulla costruzione delle formule di $L$. Siccome la semantica intende conformarsi al principio di composizionalit\`a si pu\`o pensare di specificare 
schematicamente le condizioni di verit\`a delle formule atomiche di $L$ come dipendenti dalle denotazioni dei costituenti subenunciativi e poi indicare come il valore semantico delle fomule comonenti 
determina quello dei composti vero-funzionali. Tuttavia formule quantificate di $L$ non sono formule atomiche e non sembrano propriamente composti vero-funzionali e potrebbero essere intese come 
abbreviazioni di congiunzioni e disgiunzioni. Se il dominio dell'interpretazione \`e infinito queste operazioni dovrebbero contenere infiniti congiunti e simili formule infinitarie. Un'altra considerazione va fatta 
sulle regole di formazione per $L$ in quanto permettono di costruire formule aperte che non sono propriamente enunciati. Le due ultime considerazioni sono risolte mediante la nozione di soddisfacimento di 
una formula rispetto a una data assegnazione di un valore alle variabili in modo da esplicitare anche le condizioni di verit\`a di formule quantificate che saranno vere solo se la loro condizione \`e soddisfatta da 
ogni e da almeno una assegnazione di valore a $x$. I quantificatori vengono visti come una sorta di istruzione per valutare assegnazioni di valori alle variabili che li quantificano. Si generalizza il concetto di 
soddisfacimento a tutte le formule, sia quelle aperte che agli enunciati. Si indica con $\models$. Si pu\`o ora scrivere la definizione induttiva. Si indica $\mathcal{M}^a\models\alpha$ a indicare che la formula $
\alpha$ di $L$ \`e soddisfatta nel modello $\mathcal{M}=<U, i>$ e rispetto all'assegnazione $a$ di un dato valore alle variabili di $L$ entro quella interpretazione. Con $a[x/u]$ si indica l'assegnazione 
identica ad $a$ tranne per il fatto che assegna come valore alla variabile $x$ l'individuo $u$ del dominio $U$. 
\begin{itemize}
\item Base induttiva: riguarda le formule atomiche: $\mathcal{M}^a\models P(t_1, \dots, t_n)$ se e solo se $<\mathcal{M}^a(t_1), \dots, \mathcal{M}^a(t_n)>\in\mathcal{M}(P)$. Questa clausola pu\`o essere 
interpretata come: una qualunque formula atomica $P(t_1, \dots, t_n)$ di $L$ \`e soddisfatta nell'interpretazione nel modello $\mathcal{M}=<U, i>$ e rispetto all'assegnazione $a$ se e solo se la ennupla 
ordinata di individui rispetto a questa assegnazione appartiene all'insieme denotato dal predicato $P$, equivalentemente: se e solo se fra gli individui di $U$ che sono le denotazioni dei termini $t_1, \dots, 
t_n$ in questa interpretazione e secondo questa assegnazione sussiste la relazione che \`e la denotazione del predicato ennario $P$.
\item Passo induttivo: vengono stabilite le clausole per il soddisfacimento di formule composte:
\begin{itemize}
\item $\mathcal{M}^a\models\neg\alpha$ se e solo se non $\mathcal{M}\models\alpha$.
\item $\mathcal{M}^a\models\alpha\land\beta$ se e solo se $\mathcal{M}\models\alpha$ e $\mathcal{M}\models\beta$.
\item $\mathcal{M}^a\models\alpha\lor\beta$ se e solo se $\mathcal{M}\models$ o $\mathcal{M}\models$.
\item $\mathcal{M}^a\models\alpha\Rightarrow\beta$ se e solo se non $\mathcal{M}\models\alpha$ o $\mathcal{M}\models\beta$.
\item $\mathcal{M}^a\models\alpha\Leftrightarrow\beta$ se e solo se ($\mathcal{M}\models\alpha$ e $\mathcal{M}\models\beta$ oppure non $\mathcal{M}\models\alpha$ e non $\mathcal{M}
\models\beta$.
\item $\mathcal{M}^a\models\forall x\alpha$ se e solo se per ogni $u\in U\mathcal{M}^{a[x/u]}\models\alpha$.
\item $\mathcal{M}^a\models\exists x\alpha$ se e solo se per qualche $u\in U\mathcal{M}^{a[x/u]}\models\alpha$.
\end{itemize}
\end{itemize}
La definizione di soddisfacimento nella semantica tarskiana viene strutturata in modo tale che per il soddifacimento di una formula $\alpha$ rispetto a un'assegnazione il valore attribuito da quell'assegnazione 
a variabili che non compaiono libere in $\alpha$ \`e irrilevante. In conseguenza di ci\`o gli enunciati non contenendo variabili libere sono soddisfatti rispetto a tutte le assegnazioni o rispetto a nessuna.  La 
verit\`a di una formula in un'interpretazione \`e definita come soddisfacimento rispetto a tutte le assegnazioni. Per indicare che $\alpha$ \`e vera nell'interpretazione nel modello $\mathcal{M}$ si scrive $
\mathcal{M}\models\alpha$. La definizione induttiva della relazione di soddisfacimento \`e conforme alla convenzione $V$. Consente infatti di esplicitare le condizioni di verit\`a di formule semplici e delle loro 
derivate. Poich\`e se ne possono dedurre tutti i condizionali \`e in grado di caratterizzare la verit\`a per il linguaggio predicativo $L$. 
\subsection{Verit\`a logica e conseguenza logica}
La verit\`a degli enunciati \`e caratterizzata relativamente al modello e all'interpretazione: una formula potr\`a essere vera o falsa dipendentemente da essa. Si consideri la totalit\`a delle interpretazioni possibili: 
se una formula \`e soddisfatta rispetto ad un'assegnazione in almeno un'interpretazione si dice soddisfacibile, me \`e vera rispetto a tutte le assegnazioni in almeno un'interpretazione si dice verificabile. Si pu\`o 
dare il caso che $\alpha$ sia soddisfatta rispetto a tutte le assegnazioni in tutte le interpretazioni, ovvero che sia vera in tutti i modelli: dato un qualsiasi modello $\mathcal{M}$, $\mathcal{M}\models\alpha$. 
In questo caso dice che $\alpha$ \`e una formula logicamente vera o universalmente valida, ovvero \`e una legge logica e si indica con $\models\alpha$. Le formule predicative che sono leggi logiche si dicono 
leggi logico-predicative. Attraverso la semantica formale si pu\`o caratterizzare precisamente e generalmente la nozione di legge logica: essa \`e una formula che resta vera qualunque significato si assegni ai 
suoi simboli descrittivi e si tratta di una verit\`a logica. Si pu\`o inoltre caratterizzare una conseguenza logica: si \`e detto che una formula \`e conseguenza logica di altre formule se e solo se in qualsiasi 
circostanza in cui tutte queste formule sono vere anche quella \`e vera. Essendo un modello la rappresentazione formale di una circostanza si pu\`o dire che $\beta$ \`e conseguenza logica di $\alpha_1, \dots, 
\alpha_n$ se e solo se per ogni modello $\mathcal{M}$ se $\mathcal{M}\models\alpha_1$ e $\dots$ e $\mathcal{M}\models\alpha_n$ allora $\mathcal{M}\models\beta$, ossia se e solo se tutti i modelli che 
rendono vera ciascuna delle premesse rendono vera anche la conclusione. Si pu\`o estendere la notazione in modo da poter scrivere $\alpha_1, \dots, \alpha_n\models\beta$. Un'altra notazione caratterizzabile 
\`e l'equivalenza logica. $\alpha$ e $\beta$ si dice che sono logicamente equivalenti se ciascuna delle due \`e conseguenza logica dell'altra, ovvero se tutti i modelli che rendono vera l'una rendono vera anche 
l'altra e viceversa. A differenza dei teoremi e di derivabilit\`a logica puramente sintattici leggi logiche e conseguenze logiche sono nozioni semantiche in quanto nella loro caratterizzazione si fa ricorso esplicito 
alla nozione di verit\`a. Che i teoremi del calcolo dei predicati classico coincidano con le leggi logiche \`e vero ma non scontato. 
\subsection{La verit\`a \`e inesprimibile}
La condizione formale di adeguatezza di una definizione della verit\`a per un linguaggio \`e che non consenta di dedurre contraddizioni, che \`e legata alla distinzione fra linguaggio-oggetto e metalinguaggio: se i 
due vengono confusi $V$ pu\`o generare paradossi che accadono se si assume che il predicato di verit\`a per un linguaggio sia esprimibile o definibile nello stesso linguaggio. Questo \`e dovuto al paradosso del 
mentitore $(M)$: l'enunciato $(M)$ \`e falso. Questo enunciato \`e autoreferenziale. Ci si chiede quale sia il suo valore di verit\`a e si ragiona per casi: se sia vero allora per ci\`o che dice \`e falso, supponendo sia 
falso allora \`e vero. Accettando il principio di bivalenza ciascuna di queste alternative produce una situazione paradossale. Mediante una particolare tecnica \`e possibile costruire enunciati autoreferenziali 
attraverso un linguaggio formale come $L$: essendo le formule di $L$ insiemi di oggetti possono venir prese come dominio di una interpretazione di $L$ per cui esister\`a un'interpretazione detta morfologica 
in cui formule di $L$ parlano delle stesse formule di $L$. Sotto certe condizioni \`e posibile costruire enunciati autoreferenziali mediante la diagonalizzazione che permette di associare a $\alpha[x]$ l'enunciato 
$\beta$ che si ottiene sostituendo al variabile libera con il suo nome in quella interpretazione $\beta\Leftrightarrow\alpha[x."\beta"]$ $\beta$ viene detto punto fisso di $\alpha[x]$. Si ponga ora che $V$ sia il 
predicato di verit\`a per $L$ e che sia esprimibile in $L$. Ora diagonalizzando $V(x)$ si potr\`a avere $\mu\Leftrightarrow\neg V("\mu")$ che dice che questo enunciato non \`e vero che rompe il principio di 
bivalenza. Una conclusione \`e che il predicato di verit\`a per un linguaggio $L$ non deve essere esprimibile nello stesso linguaggio $L$, secondo una versione del teorema di Tarski di indefinibilit\`a della 
verit\`a. Da qui l'importanza per cui la teoria in cui viene definita la verit\`a per $L$ sia formulata in un metalingauggio $L_1$ distinto dal linguaggio-oggetto $L$ e nel quale si possa soltanto parlare dei concetti 
semantici riguardanti $L$. Una conseguenza di questa situazione \`e che una caratterizzazione universale della verit\`a \`e impossibile. Questo metodo inoltre non sembra applicabile al linguaggio naturale in 
quanto non \`e strutturato secondo una gerarchia di metalinguaggi n\`e che ci possano essere concetti inesprimibili in esso. 