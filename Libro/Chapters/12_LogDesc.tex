\chapter{Logiche descrittive}
Le logiche descrittive sono una famiglia di logiche che permettono di modellare relazioni rispetto entit\`a in un dominio di interesse. Esistono tre tipi di entit\`a: concetti, ruoli e nomi individuali che 
rappresentano rispettivamente insiemi di individui, relazioni binarie tra individui e singolo individui nel dominio. A differenza di un database consiste in un insieme di predicati detti assiomi ognuno dei quali deve
essere vero nella situazione che si vuole descrivere. Generalmente catturano una conoscenza parziale. Possono essere divisi in tre categorie: assiomi assertivi ($ABox$), assiomi terminologici $TBox$ e 
assiomi relazionali $RBox$.
\subsubsection{Asserire fatti attraverso assiomi $\mathbf{ABox}$}
Gli assiomi $ABox$ catturano conoscenza riguardo individui nominati, ovvero a quale concetto appartengono (asserzione di concetto) e come sono messi in relazione tra di loro (asserzione di ruolo). I nomi 
individuali cos\`i caratterizzati si dicono istanze dei concetti e delle relazioni. Essendo che questa logica non fa una unique name assumption nomi diversi possono riferirsi allo stesso individuo come pu\`o 
accadere combinando conoscenza riguardo lo stesso dominio da fonti diverse, facendo ovvero ontology alignment.
\subsubsection{Asserire conoscenza terminologica dagli assiomi $\mathbf{TBox}$}
Gli assiomi $TBox$ descrivono relazioni tra concetti attraverso inclusione $\sqsubseteq$ detta anche sottoassunzione o equivalenza $\equiv$. 
\subsubsection{Modellare relazioni tra ruoli attraverso assiomi $\mathbf{RBox}$}
Gli assiomi $RBox$ si riferiscono a propriet\`a dei ruoli. Riguardo ai concetti vengono supportate inclusioni ed equivalenze. Si possono comporre ruoli attraverso $\circ$ e ruoli disgiunti attraverso
$Disjoint(A,B)$. Gli assiomi includono anche caratteristiche di ruolo come riflessivit\`a, transitivit\`a e simmetria.
\section{Costruttori per concetti e ruoli}
Gli assiomi tipici sono limitati, per descrivere situazioni pi\`u complesse si renda necessaria la creazione di nuovi concetti e ruoli utilizzando una variet\`a di costruttori. Nel caso dei concetti i costruttori possono
essere separati in basici booleani, restrizioni d
i ruolo e nominali, enumerazioni.
\subsection{Costruttori di concetti booleani}
Questi costruttori mettono a disposizione operazioni booleane classiche in vicina relaizone con le operazioni di intersezione, unione e complemento o le analoghe operazioni logiche. L'inclusione di concetti 
permette di congiungere concetti attraverso l'operazione di intersezione $\sqcap$ di altri due concetti. L'unione invece permette di descrivere un nuovo concetto attraverso l'unione di altri due con l'operatore
$\sqcup$. Il complementare di un concetto, ovvero la sua negazione \`e espresso dall'operatore $\neg$. Si definisce $\top$ il top concept, ovvero il concetto in cui ogni individuo \`e una sua istanza e $\bot$ il 
bot concept, la contraddizione, il concetto in cui nessun individuo \`e una sua istanza. 
\subsection{Restrizioni di ruolo}
$TBox$ e $RBox$ possono essere utilizzate per esprimere relazioni tra ruoli e concetti. Essendo $C$ un concetto e $R$ una relazione si definisce l'operatore di restrizione esistenziale $\exists R.C$ che 
descrive che esiste $y$ tale che $C(y)$ tutte le $x$ tali che $R(y, x)$. Un ulteriore operatore \`e la restrizione universale $\forall R.C$ che definisce che per ogni $y$, se $R(x,y)$ allora $C(y)$. Queste due
restrizioni sono utili in combinazione con $\top$ per esprimere restrizioni di dominio e range per i ruoli. Si pu\`o inoltre aggiungere a questi operatori le restrizioni almeno $\ge n$ e al pi\`u $\le n$. Inoltre
la riflessivit\`a locale espressa da $op R.Self.C$ pu\`o esprimere i concetti in relazione con s\`e stessi nel ruolo.
\subsection{Nominali}
Si possono definire concetti anche enumerando le loro istanze. Questo pu\`o essere fatto attraverso i nominali, concetti che hanno esattamente un'istanza. Combinando questi elementi atraverso l'unione nasce 
l'enumerazione. 
\subsection{Costruttori di ruolo}
Esistono pochi costruttori per i ruoli complessi. Come per esempio i ruoli inversi. Il ruolo inverso di $R: C\rightarrow C'$ viene indicato con $R^-: C'\rightarrow C$ . Viene definito il ruolo universale $U$ che 
mette in relazione tutte le paia di individui. Si pu\`o definire un ruolo $R$ come vuoto attraverso l'assioma $\top\sqsubseteq\neg\exists R.\top$. 
\subsection{Caratteristiche di ruolo}
La transitivit\`a \`e una forma di composizione di ruolo complessa, la riflessivit\`a l'equivalenza rispetto all'intersezione.
\section{La logica descrittiva $\mathbf{\mathcal{SROIQ}}$}
Formalmente ogni ontologia delle logiche descrittive \`e basata su tre insiemi finiti di simboli: un insieme $N_i$ di nomi individuali, un insieme $N_C$ di nomi di concetti e un insieme $N_R$ di nomi di ruoli.
Questi concetti sno assunti fissati per qualche applicazione e non menzionati esplicitamente. L'insieme delle espressioni di ruoli di $\mathcal{SROIQ}$ $R$ \`e definito come $R:=U|N_R|N_{R^-}$ dove
$U$ \`e il ruolo universale.  L'insieme dei concetti $C$ verr\`a definito come $C:=C_C|(C\sqcap C)|(C\sqcup C)|\neg C|\top|\bot|\exists R.C|\forall R.C|\ge n R.C|\le n R.C| \exists R.Self|\{N_i\}$. Utilizzando 
questi insiemi gli assiomi di $\mathcal{SROIQ}$ possono essere definiti come:
\begin{center}
\begin{tabular}{|c c c c c|}
\hline
$ABox$: & $C(N_i)$ & $R(N_i, N_i)$ & $N_i\approx N_i$ & $N_i\not\approx N_i$\\
\hline
$TBox$: & $C\sqsubseteq C$ & $C\equiv C$ &  & \\
\hline
$RBox$: & $R\sqsubseteq C$ & $R\equiv R$ & $R\circ R\sqsubseteq R$ & $Disjoint(R,R)$\\
\hline
\end{tabular}
\end{center}
L'ontologia o base di conoscenza \`e un insieme di tali assiomi. Per garantire l'esistenza di algoritmi di ragionamento corretti e che terminano si devono aggiungere aggiuntive restrizioni sintattiche che si 
riferiscono all'intera struttura dell'ontologia e sono pertanto chiamate restrizioni strutturali. Le due condizioni rilevanti per $\mathcal{SROIQ}$ sono basate sulla nozione di semplicit\`a e regolarit\`a, entrambe
automaticamente soddisfatte in mancanza di assiomi di inclusione di ruolo complessi. Un ruolo $R$ in un'ontologia $O$ \`e detto non semplice se qualche assioma di inclusione di ruolo complesso in $O$ 
implica istanze di $R$:
\begin{itemize}
\item Se $O$ contiene un'assioma $S\circ T\sqsubseteq R$ allora $R$ \`e non semplice.
\item Se $R$ \`e non semplice allora la sua inversa $R^-$ \`e non semplice.
\item Se $R$  \`e non semplice e $O$ contiene uno qualunque degli assiomi $R\sqsubseteq S$, $S\equiv R$ o $R\equiv S$ allora anche $S$ \`e non semplice.
\end{itemize}
Tutti gli altri ruoli sono semplici. Per un'ontologia $\mathcal{SROIQ}$ \`e richiesto che i seguenti assiomi e concetti contengano solamente ruoli semplici:
\begin{itemize}
\item Assiomi limitati: $Disjoint(R,R)$.
\item Espressioni di concetto limitate: $\exists R.Self$, $\ge n R.C$, $\le n R.C$.
\end{itemize}
L'altra restrizione strutturale \`e chiamata regolarit\`a e riguarda unicamente gli assiomi per $RBOx$. Questa restrizione si assicura che dipendenze cicliche tra assiomi di inclusione di ruoli complessi avvengano
unicamente in una forma limitata. 
\section{Semantiche delle logiche descrittive}
Il significato formale degli assiomi \`e dato dalla semantica della teoria del modello. La semantica specifica cos'\`e la conseguenza logica di un ontologia. Un'ontologia descrive una particolare 
situazione in un dominio del discorso dato, senza specificarla completamente. Dovendo gestire questa incompletezza considerando tutte le situazioni possibili dove gli assiomi di un'ontologia siano soddisfatti.
Questa caratteristica \`e detta Open World Assumption in quanto lascia le informazioni non specificate aperte a interpretazione. Una conseguenza logica di un'ontologia \`e un assioma che vale in tutte le 
interpetazioni che soddisfano l'ontologia. Un'ontologia pu\`o non essere soddisfatta da nessuna interpretazione ed \`e pertanto detta inconsistente e ogni assioma rimane indecidibile. Un'interpretazione
$I$ consiste di un insieme $\Delta^I$ detto dominio di $I$ e una funzione di interpretazione $\cdot^I$ che mappa ogni concetto atomico $A$ a un insieme $A^I\subseteq\Delta^I$, ogni ruolo atomico $R$ a
una relazione binaria $R^I\subseteq\Delta^I\times\Delta^I$ e ogni nome individuale $a$ a un elemento $a^I\in\Delta^I$. L'interpretazione di concetti e ruoli complessi segue dall'interpretazione delle entit\`a
base. 