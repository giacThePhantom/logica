\chapter{Logica descrittiva}
Le logiche descrittive sono una famiglia di logiche che permettono di parlare riguardo un dominio composto da un insieme di oggetti organizzati in concetti e messi in relazione attraverso relazioni binarie. 
\section{Knowledge Graphs}
Queste logiche permettono di costruire predicati riguardo grafi diretti etichettati chiamati Knowledge graph (KGs), i cui nodi rappresentano le classi di oggetti reali e gli archi le propriet\`a che intercorrono tra paia 
di oggetti.
\subsection{Propriet\`a}
\begin{itemize}
\item Lo schema del grafo non \`e fisso e pu\`o evolvere.
\item Non \`e necessario conoscere tutti i valori dei dati per tutti gli oggetti.
\end{itemize}
\subsection{Livelli nei Knowledge graph}
\subsubsection{Livello di schema}
\begin{itemize}
\item Tipi di entit\`a (etypes): descrizioni dei tipi di oggetti contenuti nei nodi.
\item Propriet\`a:
\begin{itemize}
\item Attributi (data properties): relazioni che associano ad ogni oggetto un tipo.
\item Relazioni (Object properties): relazioni che sussistono tra gli oggetti.
\end{itemize}
\end{itemize}
\subsubsection{Livello dei dati}
\begin{itemize}
\item Entit\`a: associano valori ai tipi di entit\`a.
\item Valori di propriet\`a:
\begin{itemize}
\item Valori di attributi: associano valori agli attributi della oggetti.
\item Valori di Relazioni: associano relazioni a valori degli oggetti.
\end{itemize}
\end{itemize}
\section{Caratterizzazione di una logica descrittiva}
Una logica descrittiva \`e caratterizzata da quattro elementi:.
\subsubsection{Linguaggio}
Un linguaggio descrittivo $L$ che determina come formare concetti (i nodi dei KG) e i ruoli (gli archi del KG e i constraints). \`E pertanto la formalizzazione di etypes, entit\`a, propriet\`a e valori di propriet\`a.
\subsubsection{Specificare conoscenza}
Un meccanismo per specificare la conoscenza riguardante etypes (concetti) e propriet\`a (ruoli), come per esempio una $TBox$, la formalizzazione dello schema del KG con etypes e properties con i constraints.
\subsubsection{Propriet\`a degli oggetti}
Un meccanismo per specificare le propriet\`a degli oggetti o entit\`a come un $ABox$, una formalizzazione delle entit\`a e valori di propriet\`a del KG.
\subsubsection{Servizi di ragionamento}
Un insieme di servizi di ragionamento che permettono di inferire nuove propriet\`a su concetti, ruoli e oggetti che sono le conseguenze logiche di quelle asserite esplicitamente nelle $TBox$ e $ABox$. Sono 
equivalenti al ragionamento logico, ovvero un estensione delle formalizzazione del ragionamento KG.
\section{Architettura di un sistema di ragionamento di una logica descrittiva}
La logica descrittiva \`e una famiglia di formalismi $KR$, dove rappresentazione e ragionamento puntano a un sistema formato da $TBox$ e $ABox$. Esiste un alfabeto di simboli formato da 	quello della 
logica proposizionale con l'aggiunta di:
\begin{itemize}
\item $\forall R$ value restriction.
\item $\exists R$ existential quantification.
\end{itemize}
E $R$ sono nomi di ruolo atomici. Si pu\`o dire che la base della conoscenza \`e formata da una $TBox$ che garantisce una conoscenza terminologica e un $ABox$ che fornisce conoscenza riguardante gli 
oggetti.
\section{Sintassi TBox - ALC (AL con il concetto di negazione)}
Regole di formazione:
\begin{itemize}
\item $< Atomic>:=A|B|\dots|P|Q|\dots|\bot|T$.
\item $<wwf>:=<Atomic>|\neg<wwf>|<wwf>\sqcap<wwf>|<wwf>\sqcup<wwf>$ 
\end{itemize}
\subsection{Interpretazione $\mathbf{AL*}$}
La funzione di interpretazione $I:<wwf>\rightarrow \Delta$.
\begin{itemize}
\item $I(\bot)=\emptyset$ e $I(T)=\Delta$ (dominio pieno universo).
\item Per ogni nome concettuale $A$ di $L$, $I(A)\subseteq\Delta$.
\item $I(\neg C)=\Delta\backslash I(C)$.
\item $I(C\sqcap D)=I(C)\cap I(D)$.
\item $I(C\sqcup D)=I(C)\cup I(D)$.
\item Per ogni nome di ruolo di $R$, $I(R)\subseteq \Delta\times\Delta$.
\item $I(\forall R.C)=\{a\in\Delta| per\ ogni\ b,\ se\ (a,b)\in I(R)\ allora\ b\in I(C)\}$.
\item $I(\exists R.T)=\{a\in \Delta| esiste\ b\ tale\ che\ (a,b)\in I(R)\}$.
\item $I(\exists R.C)=\{a\in\Delta| esiste\ b\ tale\ che\ (a,b)\in I(R),b\in I(C)$.
\item Aggiunte esterne a $ALC$:
\begin{itemize}
\item $I(\le nR)=\{a\in\Delta| |\{b|(a,b)\in I(R)\}|\le n\}$.
\item $I(\ge nR)=\{a\in\Delta| |\{b|(a,b)\in I(R)\}|\ge n\}$.
\end{itemize}
\end{itemize}
\subsection{Assiomi terminologici}
Vengono introdotti due nuovi simboli logici:
\begin{itemize}
\item $\sqsubseteq$ (sottoassunzione) con $I\models C\sqsubseteq D$ se e solo se $I(C)\subseteq I(D)$.
\item $\equiv$ (equivalenza) con $I\models C\equiv D$ se e solo se $I(C)=I(D)$:
\end{itemize}
\subsubsection{Assioma di inclusione}
$C\sqsubseteq D$ deve essere letto $C$ \`e sottoassunto da (pi\`u specifico, meno generale rispetto a) $D$. Lo stesso concetto pu\`o essere espresso come $D\sqsupseteq C$.
\subsubsection{Assioma di uguaglianza}
$C\equiv D$ deve essere letto come $C$ \`e equivalente a $S$. Vale se e solo se valgono insieme $C\sqsubseteq D$ e $D\sqsubseteq C$
\subsubsection{Definizione}
Una definizione \`e un'equivalenza con un concetto atomico sulla parte sinistra.
\subsubsection{Specializzazione}
Una specializzazione \`e un'inclusione con un concetto atomico sulla parte sinistra.
\subsubsection{Terminologia}
Una terminologia o $TBox$ \`e un insieme di definizioni e specializzazioni. Gli assiomi terminologici esprimono constraints sui concetti del linguaggio limitando cos\`i i modelli possibili. Il $TBox$ \`e l'insieme
di tutti i constraints sul modello possibile.
\subsection{TBox - ragionamento}
Date due proposizione di classe $P$ e $Q$ si presentano i seguenti problemi di ragionamento:
\begin{itemize}
\item Soddisfacibilit\`a: $T\models P$.
\item Sottoassunzione: $T\models P\sqsubseteq Q$ o $T\models Q\sqsubseteq P$.
\item Equivalenza: $T\models P\sqsubseteq Q$ e $T\models Q\sqsubseteq P$.
\item Disgiuntezza: $T\models P\sqcap Q\sqsubseteq\bot$.
\end{itemize}
\subsubsection{Soddisfacibilit\`a}
Un concetto $P$ \`e soddisfacibile rispetto a una terminologia $T$ se esiste almeno un'interpretazione $I$ con $I\models\theta$ per ogni $\theta\in T$ tale che $I\models P$ ovvero $I(P)\neq\emptyset$. In
questo caso si dice che $I$ \`e un modello per $P$. In altre parole l'interpretazione $I$ non solo soddisfa $P$ ma soddisfa tutti i constraints in $T$.
\subsubsection{Sottoassunzione}
Sia $T$ una $TBox$. Un problema di sottoassuzione rispetto a $T$ \`e definito come $T\models P\sqsubseteq Q(P\sqsubseteq_T Q)$. Un concetto $P$ \`e sottoassunto da un concetto $Q$ con rispetto a $T$
se $I(P)\subseteq I(Q)$ per ogni modello $I$ di $T$. Questo problema si riduce a derivazione logica e validit\`a quando $T$ \`e vuota.
\subsubsection{Equivalenza}
Sia $T$ una $TBox$. Un problema di equivalenza rispetto a $T$ \`e definito come $(T\models P\equiv Q)(P\equiv_T Q)$. Due concetti $P$ e $Q$ sono equivalenti rispetto a $T$ se $I(P)=I(Q)$ per ogni 
modello $I$ di $T$.
\subsubsection{Disgiuntezza}
Sia $T$ una $TBox$. Un problema di disgiuntezza rispetto a $T$ \`e definito come $T\models P\sqcap Q\sqsubseteq \bot(P\sqcap Q\sqsubseteq_T\bot)$. Due concetti $P$ e $Q$ sono disgiunti rispetto a 
$T$ se la loro intersezione \`e vuota, $I(P)\cap I(Q)=\emptyset$ per ogni modello $I$ di $T$.
\section{ABox}
\subsection{Sintassi}
Il secondo componente della base della conoscenza \`e la descrizione del mondo o $ABox$. Un'$ABox$ riguarda individui con nomi e asserisce propriet\`a riguardo essi. Si denotano nomi individuali con lettere
minuscoli e un asserzione con concetto $C$ \`e chiamata asserzione di concetto nella forma $C(a)$.
\subsection{Semantica}
Si d\`a semantica alle $ABoxes$ estendendo l'interpretazione ai nomi individuali. Un'interpretazione $I:L\rightarrow\Delta'$ non mappa concetti atomici a insiemi ma mappa ogni nome individuale a un 
elemento di $\Delta'$. ovvero $I(a)=a'\in\Delta'$. Con la Unique name assumption (UNA) si assume che nomi individuali distinti denotano oggetti distinti nel dominio. 
\subsection{Individui nella TBox}
A volte pu\`o essere conveniente permettere nomi individuali detti nominali nelle $TBox$. Sono usati per costruire concetti come per esempio con un costruttore di insieme che definisce un concetto senza 
dargli un nome enumerando i suoi elementi con $I(\{a_1,\dots,a_n\})=\{a'_1,\dots,a'_n\}$.
\subsection{Problemi di ragionamento}
\begin{itemize}
\item Soddisfacibilit\`a, consistenza: Un $ABox$ $A$ \`e consistente rispetto a $T$ se esiste almeno un'interpretazione $I$ che \`e modello sia di $A$ che di $T$.
\item Controllo delle istanze: controllare se un asserzione $C(a)$ \`e derivata da un'$ABox$, ad esempio controllare se $a$ appartiene a $C$. $A\models C(a)$ se ogni interpretazione che soddisfa $A$ 
soddisfa anche $C(a)$:
\item Recupero dell'istanza: dato un concetto $C$ recuperare tutte le istanze $a$ che soddisfano $C$.
\item Realizzazione dei concetti: dato un insieme $\alpha$ di concetti e un individuo $a$, trovare i concetti pi\`u specifici $C$ tali che $A\models C(a)$.
\end{itemize}
\subsection{Ragionamento attraverso l'espansione di un ABox}
I servizi di ragionamento su un $ABox$ e una $TBox$ aciclica possono essere ridotti a controllare un'$ABox$ espansa. Si definisce l'espansione di un'$ABox$ $A$ rispetto a $T$ l'$ABox$ $A'$ che \`e 
ottenuta da $A$ sostituendo ogni asserzione di concetto $C(a)$ con l'asserzione $C'(a)$ con $C'$ l'espansione di $C$ rispetto a $T$. L'espansione $C'(a)$ di $C(a)$ \`e l'insieme dei concetti $C''(a)$ tali che
$C''(a)$ sono usati per definire $C(a)$. L'espansione $A'$ di $A$ rispetto a $T$ contiene solo primitive.
\begin{itemize}
\item $A$ \`e consistente rispetto a $T$ se e solo se la sua espansione $A'$ \`e consistente.
\item $A$ \`e consistente se e solo se $A$ \`e soddisfacibile, ovvero non contraddittoria.
\end{itemize}
\subsection{Consistenza}
Un'$ABox$ $A$ si dice consistente rispetto a $T$ se esiste un interpretazione $I$ che \`e un modello sia di $A$ che di $T$. Si dice che $A$ \`e consistente se \`e consistente rispetto alla $TBox$ vuota.
\subsection{Controllo d'istanza}
Si deve controllare se un'istanza $C(a)$ \`e derivata da un'$ABox$, ovvero se $a$ appartiene a $C$. $A\models C(a)$ se ogni interpretazione che soddisfa $A$ soddisfa anche $C(a)$. $A\models C(a)$ se e 
solo se $A\cup\{\neg C(a)\}$ \`e inconsistente. 
\subsubsection{Recupero di istanza}
Dato un concetto $C$ si devono recuperare tutte le istanze di $a$ che soddisfano $C$. Una possibile implementazione consiste nel controllare l'istanza per tutte le istanze presenti.
\subsubsection{Realizzazione dei concetti}
Dato un insieme di concetti e un individuo $a$ si devono trovare i concetti pi\`u specifici $C$ ordinati con le sottoassunzioni tali che $A\models C(a)$. \`E il problema opposto rispetto al recupero di istanza. Una
possibile implementazione \`e rappresentata nel fare un controllo di istanza per ogni concetto.
\section{World assumptions}
\subsection{Closed world assumption CWA}
Tutto ci\`o che non \`e asserito esplicitamente \`e falso, viene utilizzata nei database.
\subsection{Open world assumption OWA}
Tutti ci\`o che non \`e specificatamente asserito \`e indecidibile, viene utilizzata nelle $ABox$.
\section{Tableaux}
\subsection{Logica descrittiva come logica multimodale}
Si pu\`o pensare alla logica descrittiva come ad una logica multimodale in cui le entit\`a sono i mondi e sono capaci di ragionamento proposizionale. Una relazione \`e una relazione di accessibilit\`a, mentre
un data type \`e un'insieme di entit\`a il cui comportamento \`e conosciuto a priori e computato dal ragionatore. Un attributo \`e una relazione di accessibilit\`a a un data type. Tutti le relazioni e gli attributi che
hanno un'entit\`a nel loro dominio definiscono tutte le maniere possibili in cui quale entit\`a ha accesso alle altre entit\`a di mondo. Un etype \`e un insieme di valori di mondi che hanno lo stesso insieme di 
relazioni di accessibilit\`a. $\exists R.C$ vuol dire accessibilit\`a ad almeno un'altra entit\`a o etype $C$. $\forall R.C$ vuol dire accessibilit\`a a tutte le entit\`a di etype $C$.
\subsubsection{Mappatura del linguaggio}
\begin{itemize}
\item $\neg$ equivalente a $\neg$.
\item $\sqcap, \sqcup$ equivalente a $\land, \lor$.
\item $\sqsubseteq$ equivalente a $\Rightarrow$.
\item Le entit\`a sono mondi.
\item $TBox$ sono mondi resi espliciti nel linguaggio come variabili:
\begin{itemize}
\item Etype $C$ scritto come $C(x)$
\item Relazione $R$ scritta come $R(x,y)$.
\item $\exists R.C$ equivalente a $\diamond_RC$ e scritto come $\exists R.C(x)$.
\item $\forall R.C$ equivalente a $\Box_RC$ scritto come $\forall R.C(x)$.
\end{itemize}
\item $ABox$ sono mondi resi espliciti nel linguaggio come costanti.
\begin{itemize}
\item Etype $C$ scritto come $C(a)$
\item Relazione $R$ scritta come $R(a,b)$.
\item $\exists R.C$ equivalente a $\diamond_RC$ e scritto come $\exists R.C(a)$.
\item $\forall R.C$ equivalente a $\Box_RC$ scritto come $\forall R.C(a)$.
\end{itemize}
\end{itemize}
\subsection{L'algoritmo dei tableaux}
La formula $C$ di input \`e trasformata in $NNN$ (negation normal form) con la negazione abbassata per negare le formule atomiche in modo da evitare le regole di negazione. Una $ABox$ $A$ \`e costruita
incrementalmente secondo i constraints in $C$ identificati attraverso decomposizione seguendo delle precise regole di trasformazione. Ogni volta che si ha pi\`u di un'opzione si divide lo spazio delle soluzioni
come in un albero di decisione. Quando si \`e trovata una contraddizione si deve cercare un altro cammino nello spazio delle soluzioni. L'algoritmo si ferma quando trova una $A$ consistente che soddisfa tutti i
constraints in $C$ (soddisfacibile) o quando non esiste una $A$ consistente (la formula \`e insoddisfacibile).
\subsection{Regole di trasformazione}
\subsubsection{$\mathbf{\sqcap}$-rule}
La condizione \`e che $A$ contenga $(C_1\sqcap C_2)(x)$ ma non $C_1(x)$ e $C_2(x)$ contemporaneamente. Allora si crea $A'=A\cup\{C_1(x), C_2(x)\}$
\subsubsection{$\mathbf{\sqcup}$-rule}
La condizione \`e che $A$ contenga $(C_1\sqcup C_2)(x)$ ma n\`e $C_1(x)$ n\`e$C_2(x)$. Allora si crea $A'=A\cup\{C_1(x)\}$ e $A''=A\cup\{C_2(x)\}$.
\subsubsection{$\mathbf{\exists}$-rule}
La condizione \`e che $A$ contenga $(\exists R.C)(x)$ ma non esiste $z$ tale che sia $C(z)$ che $R(x,z)$ siano in $A$. Allora si crea $A'=A\cup\{C(z), R(x,z)\}$.
\subsubsection{$\mathbf{\forall}$-rule}
La condizione \`e che $A$ contenga $(\forall R.C)(x)$ ma non contiene $C(z)$. Allora si crea $A'=A\cup\{C(z)\}$.
\section{Tableaux}
\begin{itemize}
\item Per ogni restrizione esistenziale si introduce un nuovo individuo come un role filler e questo individuo deve soddifare il constraint espresso dalla restrizione.
\item L'algoritmo utilizza la restrizione di valore in interazione con ruoli gi\`a definiti per imporre nuovi constraints sugli indiviudi.
\item Per i constraints per le disgiunzioni si provano entrambe le possibilit\`a in ptrove successive.
\end{itemize}